\documentclass[10pt,a4paper]{article}
\usepackage[latin1]{inputenc}
\usepackage{amsmath}
\usepackage{amsfonts}
\usepackage{amssymb}
\usepackage{graphicx}
\usepackage{cite}
\usepackage{colortbl}
\usepackage{tabu}
\usepackage{fullpage}
\usepackage{fancyhdr}
\usepackage{geometry}
\geometry{margin=0.5in}
\everymath{\displaystyle}
\usepackage{lscape}

\author{Ignace Bossuyt}
\title{Handleiding differentiaalvergelijkingen}

% -aux-directory=build


\begin{document}

Bemerk dat er vier verschillende concepten zijn:

\begin{center}
	\centering
	{\tabulinesep=1.5mm
		\begin{tabu}{|c||c|} 
			\hline
integraal over kromme & lijnintegraal \\ \hline
integraal over oppervlak & oppervlakintegraal \\ \hline
	\end{tabu}}
\end{center}

Het grootste verschil zit feitelijk hier: in het ene geval inegreren we een scalaire functie. In het andere geval integreren
we in feite over een vectorveld.

\begin{enumerate}
\item kijk bijvoorbeeld naar de totale oppervlakte van een muurtje met gegeven hoogte of beschou we gemiddelde tem-
peratuur van een vlakke of gebogen plaat of het massacentrum van een voorwerp
\item we hebben een vectorveld ter beschikking. We beschouwen bijvoorbeeld de out
ow van materie doorheen een
kromme (in 2D) of doorheen een oppervlak. Het vectorveld zelf kan aan leuke eigenschappen voldoen zoals conser-
vativiteit.
\end{enumerate}

\section{Integraal over/langs kromme en integraal over oppervlak}

\begin{center}
	\centering
	{\tabulinesep=1.5mm
		\begin{tabu}{|c|c|c|} 
			\hline
\multicolumn{3}{|c|}{integraal over/langs kromme} \\ \hline
parametrisatie & $\vec{\gamma}(t)$ & $t \in [a,b]$ \\ \hline
lengte v/e kromme &\multicolumn{2}{c|}{$l(C) = \int_a^b ||\gamma'(t)|| dt$} \\ \hline
natuurlijke parametrisatie & $r(x) = \vec{\gamma}(t(s))$ & $s \in [0,1] $ \\ \hline
\cellcolor{yellow} integraal over/langs een kromme &\multicolumn{2}{c|}{$M(C) = \int_a^ f[\vec{\gamma}(t)] ||\vec{\gamma}'(t)|| dt$} \\ \hline
	& \multicolumn{2}{c|}{toepassing: berekening massacentrum} \\ \hline
	
	\end{tabu}}
\end{center}

\begin{center}
	\centering
	{\tabulinesep=1.5mm
		\begin{tabu}{|c|c|c|} 
			\hline
			\multicolumn{3}{|c|}{integraal over/langs oppervlak} \\ \hline
			parametrisatie & $\vec{\Sigma}(u,v)$ & $(u,v) \in D \in  \mathbb{R}^2$ \\ \hline
			oppervlakte v/e oppervlak &\multicolumn{2}{c|}{$A(S) = \iint_D ||\vec{n}(u,v)|| dudv$} \\ \hline
			\cellcolor{yellow} integraal over/langs een oppervlak &\multicolumn{2}{c|}{$M(S) = \iint_D f[ \vec{\Sigma}(u,v)  ] ||\vec{n}(u,v)|| dudv$}  \\ \hline
		
			& \multicolumn{2}{|c|}{toepassing: berekening massacentrum} \\ \hline
	\end{tabu}}
\end{center}

\section{Vectoranalyse: lijintegraal en oppervlakintegraal}
\begin{center}
	\centering
	{\tabulinesep=1.5mm
		\begin{tabu}{|c|c|} 
			\hline
\multicolumn{2}{|c|}{lijnintegraal} \\ \hline
kromme met parametrisatie $\vec{\gamma}(t)$ & $t \in [a,b]$ \\ \hline
\cellcolor{yellow} lijnintegraal in het vectorveld $\vec{F}$ & $\int_a^b \vec{F}(\vec{\gamma}(t)).\vec{\gamma}'(t)dt$ \\ \hline
	\end{tabu}}
\end{center}

\begin{center}
	\centering
	{\tabulinesep=1.5mm
		\begin{tabu}{|c|c|} 
			\hline
			\multicolumn{2}{|c|}{oppervlakintegraal} \\ \hline
			oppervlak met parametrisatie $\vec{\Sigma}(u,v)$ & $(u,v) \in D \in \mathbb{R}^2 $ \\ \hline
			\cellcolor{yellow} oppervlakintegraal in het vectorveld $\vec{F}$ & $\iint_D \vec{F}(\vec{\Sigma}(u,v)).\vec{n}(u,v)dudv$ \\ \hline
	\end{tabu}}
\end{center}


\newpage
\subsection{Vectoranalyse: enkele leuke stellingen en gelijkheden}


\begin{center}
	\centering
	{\tabulinesep=1.5mm
		\begin{tabu}{|c|c|} 
			\hline
rotor van een vectorveld $\vec{F}$ & $\vec{\nabla} \times \vec{F}$ \\ \hline

divergentie van een vectorveld $\vec{F}$ & $\vec{\nabla}.\vec{F}$ \\ \hline \hline

gradi\"ent van een scalaire functie $f$ & $\vec{\nabla}f$ \\ \hline

Laplaciaan van een scalaire functie $f$ & $\Delta f = \vec{\nabla}^2 f = \vec{\nabla}.\vec{\nabla}f$ \\ \hline
	\end{tabu}}
\end{center}


Voor een \textit{conservatief vectorveld} gelden volgende eigenschappen
\begin{itemize}
	\item heeft potentiaalfunctie $f$
	\item is irrotationeel, namelijk $\vec{\nabla} \times \vec{F} = 0$
\end{itemize}

Enkele belangrijke stellingen

\begin{center}
	\centering
	{\tabulinesep=1.5mm
		\begin{tabu}{|c|c|} 
			\hline
stelling & \\ \hline \hline

Green & dubbele integraal over gebied $D$ \\
(2D-vectorveld $\vec{F}$)	& vs. lijnintegraal over kromme $C$ \\ 
 	& $\iint_D \left[ \frac{\partial N}{\partial x } - \frac{\partial M}{\partial y} \right] dx dy = \int_C Mdx + Ndy $ \\ 
	& toepassing: oppervlakteberekening \\ \hline

Stokes & oppervlakintegraal over gebied $S$ \\
	& vs. lijnintegraal over rand $C$ (kromme) \\ 
(2D-vectorveld $\vec{F}$) & $\iint_S \vec{\nabla} \times \vec{F} = \int_c \vec{F}$ \\ \hline

Gauss & integraal over 3D gebied \\
	& integraal over zijn rand \\
(3D-vectorveld $\vec{F}$) & $\iiint_d \vec{\nabla}. \vec{F} = \iint_S \vec{F}$ \\ 

(2D-vectorveld $\vec{F}$) & $\iint_D \vec{\nabla} . \vec{F} =  \int_S f$ \\ 
potentiaalfunctie $f$ & \\ \hline
	
	\end{tabu}}
\end{center}


Bij het opstellen van dit overzicht werd gebruik gemaakt van \cite{VandewalleStefan2017AIS}. Dank aan meneer Scheerlinck voor het ontdekken van fouten/typo's.

\bibliography{bib}
\bibliographystyle{plain}

%\vspace*{\fill}
%(gemaakt door Ignace Bossuyt)
%\end{landscape}
\end{document}