\documentclass[10pt,a4paper]{article}
\usepackage[latin1]{inputenc}
\usepackage{amsmath}
\usepackage{amsfonts}
\usepackage{amssymb}
\usepackage{graphicx}
\usepackage{cite}
\usepackage{colortbl}
\usepackage{tabu}
\usepackage{fullpage}
\usepackage{fancyhdr}
\usepackage{geometry}
\geometry{margin=0.5in}
\everymath{\displaystyle}
\usepackage{lscape}


\author{Ignace Bossuyt}
\title{Handleiding differentiaalvergelijkingen}
\begin{document}

\section{Extremaalproblemen, zonder beperkingen}

\section{Extremaalproblemen met gelijkheidsbeperkingen}

\section{Extramaalproblemen met ongelijkheidsbeperkingen}

\section{Variatierekening}


\begin{center}
	\centering
	{\tabulinesep=1.5mm
		\begin{tabu}{|c|c|} 
			\hline
\multicolumn{2}{|c|}{Bepaal $y(x)$ zodat $I$ maximaal is} \\ \hline
& $I = \int_a^bF(x,y(x),y'(x),...)dx$ \\ \hline

Euler-Lagrange differentiaalvergelijking & $I = \int_a^bF(x,y(x),y'(x))dx$ \\
& met Dirichlet randvoorwaarden \\
& $y$ is oplossing van $\boxed{F_y -F_{y',x}-F_{y',y}y' - F_{y',y'}y''}$ \\ \hline


	Dirichlet randvoorwaarden & $\begin{aligned} y(a) = \alpha \\ y(b) = \beta \end{aligned}$ \\ \hline \hline
	
	natuurlijke randvoorwaarden & $y(a) = \alpha$ \\
	& eindpunt op lijn $x=b$ \\ \hline
	
	transversaliteitsvoorwaarden & $y(a)= \alpha$ \\
	& eindpunt op kromme $\psi(t)$ \\ \hline
	
	nevenvoorwaarden & $L= \int_{a}^b \psi(x,y,y') dx$ \\ \hline \hline
	\end{tabu}}
\end{center}


\begin{center}
	\centering
	{\tabulinesep=1.5mm
		\begin{tabu}{|c|c|} 
			\hline
\multicolumn{2}{|c|}{Toepassingen} \\ \hline

korste weg & \\ \hline

brachistochroon & $T = \int_0^a \sqrt{\frac{1+y'^2}{2gy}}dx$ \\ 
	* Dirichlet & $\begin{aligned} y(0) = 0 \\ y(a)= b \end{aligned}$ \\
	 * natuuurlijk & $\begin{aligned} y(0) = 0 \\ y'(a)= 0 \end{aligned}$ \\
	 *transversaliteit &  $\begin{aligned} y(0) = 0 \\ y'(a) = - \frac{1}{m} \end{aligned} $ \\ \hline

kettinglijn & \\ \hline

	\end{tabu}}
\end{center}



\newpage
Bij het opstellen van dit overzicht werd gebruik gemaakt van \cite{VandewalleStefan2017AIS}.

\bibliography{bib}
\bibliographystyle{plain}

%\vspace*{\fill}
%(gemaakt door Ignace Bossuyt)
%\end{landscape}
\end{document}