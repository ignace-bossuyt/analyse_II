\documentclass[10pt,a4paper]{article}
\usepackage[latin1]{inputenc}
\usepackage{amsmath}
\usepackage{amsfonts}
\usepackage{amssymb}
\usepackage{graphicx}
\usepackage{cite}
\usepackage{colortbl}
\usepackage{tabu}
\usepackage{fullpage}
\usepackage{fancyhdr}
\usepackage{geometry}
\geometry{margin=0.5in}

\author{Ignace Bossuyt}
\title{Handleiding differentiaalvergelijkingen}
\begin{document}

%\maketitle

\section{Differentiaalvergelijkingen}
Belangrijk: maak een onderscheid tussen lineaire en niet-lineaire differentiaalvergelijking. De oplossingsmethode verschilt en het is dus belangrijk van in het begin te kijken met welk geval we te maken hebben.

\subsection{Lineaire differentiaalvergelijking}
Voor de algemene oplossing van $Ly=f$ geldt altijd: $\displaystyle{ y_{algemeen} = y(x) = y_p + \sum_{i=1}^n c_i y_i} $ waarin $y_i$ met $i=1..n$ een fundamenteel stel vormt. 

\begin{center}
\centering
{\tabulinesep=1.5mm
\begin{tabu}{|c||c|c|} 
\hline
\multicolumn{3}{|c|}{\cellcolor{yellow} willekeurige coefficienten, eerste orde $y' = a(x)y + b(x)$} \\
\multicolumn{3}{|c|}{(scheiding der veranderlijken)} \\ \hline
$y_H = C y_1(t) =  C \exp \left(\int_{x_0}^x a(t)dt \right) $ & \multicolumn{2}{c|}{$y_P = C(x) \exp \left(\int_{x_0}^x \frac{b(t)}{y_1(t)}dt \right) $}  \\  \hline \hline

\multicolumn{3}{|c|}{\cellcolor{yellow} constante coefficienten, willekeurig orde $p(D)y=f$} \\  
homogeen & \multicolumn{2}{c|}{partikulier}  \\ \hline
$\rightarrow$ bepaal fundamenteel stel &  $f$ zelf oplossing van een of andere $p(D)y=0$ & anders \\ 
(zie Tabel 1.1 op pagina 19) & $\rightarrow$  'methode onbepaalde coefficienten' & $\rightarrow$ 'variatie van de constante' \\ 
& $f(x) = P_m(x) e^{\alpha x} \cos(\beta x) + Q_m(x) e^{\alpha x} \sin(\beta x)$ & functie van Green $K(x,t)$ \\ 
& $y_P =x^l \left[ (...) e^{\alpha x} \cos(\beta x)  + (...) e^{\alpha x} \sin(\beta x) \right]$ & $\displaystyle{y_P = \int_{x_0}^x K(x,t) \frac{f(t)}{a_0(t)}dt}$\\ \hline  \hline

\multicolumn{3}{|c|}{\cellcolor{yellow} Euler differentiaalvergelijking} \\
\multicolumn{3}{|c|}{\cellcolor{yellow} (veranderlijke coefficienten, willekeurige orde, speciaal geval)} \\ \hline



\end{tabu}}
\end{center}

\subsection{Niet-Lineaire differentiaalvergelijkingen}
Het beginwaardeprobleem $y'=f(x,y)$ met $y(x_0)=y_0$ heeft een unieke oplossing. 

\begin{center}
	\centering
	{\tabulinesep=1.5mm
		\begin{tabu}{|c||c|c|} 
			\hline
			\multicolumn{3}{|c|}{\cellcolor{yellow} Bernouilli differentiaalvergelijking $y' = a(x)y + b(x)y^{\alpha}$} \\ 
			\multicolumn{3}{|c|}{pas transformatie $y=z^{\beta}$ toe met $\displaystyle{\beta = \frac{1}{1-\alpha}}$}  \\
			\multicolumn{3}{|c|}{los de nieuwe (lineaire) differentiaalvergelijkin op} \\ \hline \hline
			
			\multicolumn{3}{|c|}{\cellcolor{yellow}  differentiaalvergelijking $Mdx + Ndy = 0$} \\ 
			(I) 'scheiding van de veranderlijken' & (II) 'exacte differentiaalvergerlijking' & (III) 'integrerende factor' \\ \hline
			(speciaal geval) & indien $\displaystyle{\frac{\partial M}{\partial y} = \frac{\partial N}{\partial x}}$ geldt $G(x,y) = C$ & indien $\displaystyle{\frac{\partial M}{\partial y} \not = \frac{\partial N}{\partial x}}$ \\
			 
			integreer $Mdx + Ndy = 0$ & ($G$ is impliciete voortelling oplossing)   & kies factor $P(x)$ \\ 
			en werk verder uit & $\rightarrow$ $G=\int Mdx + F(y)$ en $\displaystyle{\frac{\partial G}{\partial y}=N}$ & ga naar (II) \\
			& $\rightarrow$ $G=\int Ndy + F(x)$ en $\displaystyle{\frac{\partial G}{\partial yx}=M}$ & \\\hline \hline
					
			\multicolumn{3}{|c|}{\cellcolor{yellow}  oplossing via parametrisatie} \\ \hline \hline
			
			\multicolumn{3}{|c|}{\cellcolor{yellow} mogelijke toepassingen} \\
			stroomlijnen & orthogonale krommen & steilste-hellingspad \\ \hline 
			
	\end{tabu}}
\end{center}

Bij het opstellen van dit overzicht werd gebruik gemaakt van~\cite{VandewalleStefan2017AIS}.

\bibliography{bib}
\bibliographystyle{plain}

%\vspace*{\fill}
%(gemaakt door Ignace Bossuyt)
\end{document}