\documentclass[10pt,a4paper]{article}
\usepackage[latin1]{inputenc}
\usepackage{amsmath}
\usepackage{amsfonts}
\usepackage{amssymb}
\usepackage{graphicx}
\usepackage{cite}
\usepackage{colortbl}
\usepackage{tabu}
\usepackage{fullpage}
\usepackage{fancyhdr}
\usepackage{geometry}
\geometry{margin=0.5in}
\everymath{\displaystyle}
\usepackage{lscape}

\author{Ignace Bossuyt}
\title{Handleiding differentiaalvergelijkingen}
\begin{document}

\section{Reeksen van getallen (oneindige sommen)}

\paragraph{Definities}
\begin{center}
\centering
{\tabulinesep=1.5mm
\begin{tabu}{|c|c|c|c|} 
\hline
\multicolumn{2}{|c|}{reeks van reeele getallen $(\{ u_n \}_{n \geq l},\{s_n \}_{n \geq l})$} \\ \hline
termenrij & rij van partieelsommen \\ 
$\{u_n\}_{n \geq l}$ & $\{s_n\}_{n \geq l}$ \\ \hline
\multicolumn{2}{|c|}{korte notatie} \\
\multicolumn{2}{|c|}{$s_n = \sum_{k=l}^{n} u_k$} \\  \hline
\end{tabu}}
\end{center}


\begin{center}
\centering
{\tabulinesep=1.5mm
\begin{tabu}{|c|c|c|c|} 
\hline
\multicolumn{2}{|c|}{Convergentie van een reeks: convergentie van haar rij van partieelsommen.} \\ \hline
absoluut convergent & voorwaardelijk convergent \\
$\sum_{n=1}^{\infty} |u_n|$ is convergent & convergentie maar GEEN absolute convergentie \\ \hline
\end{tabu}}
\end{center}

\paragraph{Convergentietesten}

\begin{center}
\centering
{\tabulinesep=1.5mm
\begin{tabu}{|c|c|c|c|} 
\hline
\multicolumn{2}{|c|}{algemeen: criterium Cauchy} \\ \hline

convergentie & $\forall \epsilon>0, \exists N, \text{zodat} \, \forall m \geq n \geq m$ \\  
& $|u_n + u_{n+1} + \hdots + u_m| < \epsilon$ \\ 

NODIGE voorwaarde & $lim_{n \to \infty} u_n = 0$ \\ \hline

divergentie & $\exists \epsilon>0, \forall N, \text{zodat} \, \exists m \geq n \geq m$ \\ 
& $|u_n + u_{n+1} + \hdots + u_m| \geq \epsilon$ \\ \hline
\end{tabu}}
\end{center}


\begin{center}
\centering
{\tabulinesep=1.5mm
\begin{tabu}{|c|c|c|c|c|c|c|} 
\hline
\multicolumn{7}{|c|}{\textbf{reeksen met enkel positieve termen} $\sum u_n$} \\ \hline

& vergelijkingstest 
	& integraaltest %$\sum_{n=1}^{\infty} u_n$ 
	& d'Alembert (I) 
	& d'Alembert (II) 
	& Cauchy 
	& Raabe \\ 

& & & & verhoudingstest & &  \\ \hline

CONV 
	&  $\sum v_k$ conv
	&  $\int_1^{\infty} f(x) dx$
	&  $\limsup_{n\to \infty} \frac{u_{n+1}}{u_n} < 1 $
	&  $\lim_{n\to \infty} \frac{u_{n+1}}{u_n} < 1 $
	&  $\limsup_{n \to \infty} \sqrt[n]{u_n} < 1$
	& $c>1$
	\\ 
	
	& $u_n \leq v_n$ 
	& conv
	&
	&
	& 
	& $c= \lim_{n \to \infty} n \left( 1 - \frac{u_{n+1}}{u_n} \right)$
	\\ \hline

DIV 
	& $\sum v_k$ div
	& anders 
	& $\liminf_{n\to \infty} \frac{u_{n+1}}{u_n} > 1 $
	& $\lim_{n\to \infty} \frac{u_{n+1}}{u_n} < 1 $
	& $\limsup_{n \to \infty} \sqrt[n]{u_n} > 1$
	& $c<1$
	\\ 

	& $v_n \leq u_n$ 
	&
	&
	&
	& & 
	\\ \hline

\multicolumn{7}{|c|}{\textbf{reeksen met enkel negatieve termen}, idem op tegengestelde na} \\  \hline

\multicolumn{7}{|c|}{\textbf{wisselreeksen}, bv. Leibnizreeks} \\ \hline

\end{tabu}}
\end{center}

\newpage
\paragraph{Enkele memorabele reeksen}
\begin{center}
\centering
{\tabulinesep=1.5mm
\begin{tabu}{|c|c|c|c|} 
\hline
& & & \\ \hline
reeks van Grandi 
	& $1-1 + 1-1 + \hdots $ 
	& divergent 
	& rij van partieelsommen $1,0,1 \hdots $ heeft geen limiet \\ \hline
	
harmonische reeks 
	& $\sum_{n=1}^{\infty} \frac{1}{n}$ 
	& divergentie 
	& voldoet niet aan Cauchy criterium \\ \hline
	
meetkundige reeks, \textit{reden} $q$ 
	& $\sum_{n=0}^{\infty} q^k$ 
	& $|q| < 1$: convergentie & $s_n = \frac{1}{1-q}$ \\ 
	
	& & $|q| >1$: divergentie 
	& niet voldaan aan $\lim_{n\to \infty} u_n = 0$ \\ \hline
	
Dirichletreeks 
	& $\sum_{n=1}^{\infty} n^{-p}$ 
	& $p >1$: convergentie 
	& $p=2$: $s_n = \frac{\pi^2}{6}$ \\ 
	
	& & & $p=4$: $s_n = \frac{\pi^4}{90}$ \\ 
	
	& & $p < 1$: divergentie & \\ \hline

(Riemann-zetafunctie) 
	& $\zeta(p) = \sum_{n=1}^{\infty} \frac{1}{n^p}$ 
	& met $p \in \mathbb{C}$ 
	& toepassing: getaltheorie  \\ \hline

leuke reeks (1)
	& $\sum_{n=2}^{\infty} \frac{1}{n^p \ln(n)}$ 
	& $p \leq 1$: divergentie & \\
	
	& & $p>1$: convergentie & \\ \hline

leuke reeks (2)
	& $\sum_{n=2}^{\infty} \frac{1}{n \ln^p(n)}$ 
	& $p \leq 1$: divergentie & \\
	
	& & $p>1$: convergentie & \\ \hline

\cellcolor{yellow} Leibnizreeks 
	& $\sum_{n=1}^{\infty} (-1)^{n+1} u_n$ 
	& convergent 
	& \\ 
		
	& $u_1 \geq \hdots u_1 \geq 0$  
	& & \\
	
	& $\lim_{n \to \infty} u_n = 0$ 
	& & \\ \hline

harmonische wisselreeks 
	& $\sum_{n=1}^{\infty} (-1)^{n+1} \frac{1}{n}$ 
	& convergentie 
	& $s_n = \ln(2)$ \\ \hline

\end{tabu}}
\end{center}

Bij het opstellen van dit overzicht werd gebruik gemaakt van \cite{VandewalleStefan2017AIS}.

\bibliography{bib}
\bibliographystyle{plain}

%\vspace*{\fill}
%(gemaakt door Ignace Bossuyt)
%\end{landscape}
\end{document}