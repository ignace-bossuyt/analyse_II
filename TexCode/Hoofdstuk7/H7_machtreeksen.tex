\documentclass[10pt,a4paper]{article}
\usepackage[latin1]{inputenc}
\usepackage{amsmath}
\usepackage{amsfonts}
\usepackage{amssymb}
\usepackage{graphicx}
\usepackage{cite}
\usepackage{colortbl}
\usepackage{tabu}
\usepackage{fullpage}
\usepackage{fancyhdr}
\usepackage{geometry}
\geometry{margin=0.5in}
\everymath{\displaystyle}
\usepackage{lscape}


\author{Ignace Bossuyt}
\title{Handleiding differentiaalvergelijkingen}
\begin{document}

\section{Machtreeksen}
\begin{center}
\centering
{\tabulinesep=1.5mm
\begin{tabu}{|c|c|c|} 
\hline
reeks van getallen & reeks van functies & machtreeks \\ \hline
$\sum_{n=0}^{\infty} a_n \xi^n$ & $\sum_{n=0}^{\infty} f_n(x)$ & $\sum_{n=0} ^{\infty} a_n (x-x_0)^n$ \\ \hline
\end{tabu}}
\end{center}

\begin{center}
	\centering
	{\tabulinesep=1.5mm
		\begin{tabu}{|c|c|} 
			\hline
convergentiestraal & $R = \sup \left \{  |x-x_0| \left \lvert \sum_{n=0}^{\infty} a_n(x-x_0)^n \text{convergent} \right. \right \}$ \\ 
	& $R= \frac{1}{\limsup_{n \to \infty} \sqrt[n]{|a_n|}}$ \\
	& $R = \lim_{n \to \infty} \left \lvert \frac{a_n}{a_{n+1}} \right \rvert $ \\ \hline

convergentie-interval & $(x_0-R, x_0+R)$ \\ \hline
	\end{tabu}}
\end{center}

\begin{center}
	\centering
	{\tabulinesep=1.5mm
		\begin{tabu}{|c|c|} 
			\hline
Taylor veelterm & $f(x) = P_n(x) + R_n(x)$ \\ \hline
$P_n(x) = \sum_{n=0}^{\infty}\frac{f^k(x_0)}{k!}(x-x_0)$ & $R_n(x) = \int_{x_0}^{x} \frac{(x-t)^n}{n!}f^{n+1} (t)dt$ \\ 
	& $R_n(x) = f^{(n+1)} (\xi) \frac{(x-x0)^{n+1}}{(n+1)!}$ \\ \hline
	
functie analytisch in open interval $I$ & $lim_{n \to \infty} R_n(x)= 0$ \\ \hline
	\end{tabu}}
\end{center}

\subsection{Oplossen van differentiaalvergelijkingen a.h.v. reeksontwikkeling}
\begin{center}
	\centering
	{\tabulinesep=1.5mm
		\begin{tabu}{|c|c|} 
			\hline
\multicolumn{2}{|c|}{$\begin{aligned} a_0(x) y''(x) + a_1(x)y'(x) + a_2(x)y(x) &= 0 \\ y(x_0) &= 0  \\ y'(x_0) &= y_1  \end{aligned}$} \\ \hline

rond regulier punt & $\frac{a_1(x)}{a_0(x)}$ en $\frac{a_2(x)}{a_0(x)}$ zijn analytisch in $x_0$ \\ 
	 oplossing & $y(k) = \sum_{k=0}^{\infty} C_k (x-x_0)^k$ \\
	 & recursiebetrekking \\ \hline
	
rond regulier-singulier punt &  $q(x) = (x-x_0)\frac{a_1(x)}{a_0(x)}$ en $r(x) = (x-x_0)^2\frac{a_2(x)}{a_0(x)}$ zijn analytisch in $x_0$ \\
	herschrijf differentiaalvergelijking & $(x-x)^2y''(x) + (x-x_0)q(x)y'(x) + r(x)y(x) = 0$ \\ 
	(normaalvorm Frobenius ) & \\ 
	oplossing & $y(x) = (x-x_0)^{\nu} \sum_{k=0}^{\infty} C_k (x-x0)^k$ \\
	& recursiebetrekking \\ \hline

speciaal geval: diff. verg. Bessel  & $x^2 y''(x) + xy'(x) + (x^2 - p^2)y(x) = 0$ \\ \hline

	\end{tabu}}
\end{center}


Bij het opstellen van dit overzicht werd gebruik gemaakt van \cite{VandewalleStefan2017AIS}.

\bibliography{bib}
\bibliographystyle{plain}

%\vspace*{\fill}
%(gemaakt door Ignace Bossuyt)
%\end{landscape}
\end{document}